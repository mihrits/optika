\documentclass[11pt]{article}

\usepackage[estonian]{babel}
\usepackage[utf8]{inputenc}
\usepackage{amsmath}
\usepackage{tikz}
\usepackage{graphicx, wrapfig}
\usepackage[per-mode=symbol, decimalsymbol=comma, exponent-product = \cdot]{siunitx}
\usepackage{enumerate}
\usepackage{physics}
\usepackage[margin=2.5cm]{geometry}
\usepackage{lmodern, microtype}
\usepackage{hyperref}

\setlength\parindent{0pt}

\newcommand{\yl}[1]{\noindent{\bf Ülesanne #1} \quad}
\newcommand{\lah}{\noindent{\bf Lahendus} \quad}

\begin{document}
{\small Optika (LOFY.01.008) seminar \hfill kevad 2020}\\

{\Large 12. nädal}\\

Tuletuskäikude nimekiri koos allikatega, kust tuletuskäike leida võib. \\

\begin{enumerate}
  \item Lainevõrrand lähtuvalt Maxwelli võrranditest --- õpik ptk 2.1 ja 2.2;
  \item Snelli seadus (murdumisseadus) lähtuvalt Fermat' printsiibist --- seminari ülesanne;
  \item Brewsteri nurga ja täieliku sisepeegeldumisnurga tuletus murdumisseadusest --- õpik ptk 4.5 ja 4.6;
  \item Beer-Lambert'i seaduse (neeldumisseaduse) tuletamine --- õpik ptk 4.8 (v 4.28);
  \item Stefan-Boltzmanni seaduse tuletus Plancki musta keha kiirguse põhjal --- semianri ülesanne;
  \item Wieni nihkeseaduse tuletus Plancki musta keha kiirguse seaduse põhjal --- seminari ülesanne;
  \item interferentsiliikme tuletus samasagedusega lainete liitmisel --- õpik ptk 5.2 (v 5.5);
  \item kahe tasalaine interferntsimustri perioodi tuletus, kui lained levivad optilise telje suhtes väikeste nurkade \(\alpha\) ja \(-\alpha\) all --- õpik ptk 5.2 (v 5.10)
  \item erinevate sagedustega lainete liitumine --- seminari ülesanne;
  \item Youngi katse interferentsimustri periood --- õpik ptk 5.3.1;
  \item optiline käiguvahe tasaparalleelses plaadis --- õpik ptk 5.5 (v 5.33);
  \item Netwoni rõngaste katses rõnga raadisused --- õpik ptk 5.5.1;
  \item Michelsoni interferomeetri interferentsimustri periood --- õpik ptk 5.7 ja 5.5 (samakalde  ja samapaksusinterferents);
  \item Fresneli tsoonide raadiused ümmarguselt avalt sõltuvalt tsooni järgust --- õpik ptk 6.2.2;
  \item Fraunhoferi difraktsiooni kiiritustihedus ühe pilu korral --- õpik ptk 6.4.1;
  \item N piluga difraktsioonivõre kiiritustihedus --- õpik ptk 6.5.1;
  \item murdumisnäitaja sõltuvus sagedusest --- õpik ptk 7.1.2;
  \item kiire kõrvalekalle prismat läbides --- geomeetrilise optika konspekt ptk 1.3 (lk 6-8)
  \item Doppleri efekt klassikalisel juhul --- õpik ptk 9.4 (v 9.10).

\end{enumerate}



\end{document}
