\documentclass[11pt]{article}

\usepackage[estonian]{babel}
\usepackage[utf8]{inputenc}
\usepackage{amsmath}
\usepackage{tikz}
\usepackage{graphicx, wrapfig}
\usepackage[per-mode=symbol, decimalsymbol=comma, exponent-product = \cdot]{siunitx}
\usepackage{enumerate}
\usepackage{physics}
\usepackage[margin=2.5cm]{geometry}
\usepackage{lmodern, microtype}
\usepackage{hyperref}

\setlength\parindent{0pt}

\newcommand{\yl}[1]{\noindent{\bf Ülesanne #1} \quad}
\newcommand{\lah}{\noindent{\bf Lahendus} \quad}

\begin{document}
{\small Optika (LOFY.01.008) seminar \hfill kevad 2020}\\

{\Large 12. nädala näidisülesanded seminariks}\\

Ülesannete lahendamise eest saab aktiivsuspunkte. Nende ülesannete lahendamisel saab seminari juhendaja käest vihjeid. Kui soovid vabatahtlikult mõnda ülesannet lahendada, siis anna sellest soovist Moodle'is teada enne reedet.

\rule{\linewidth}{1pt} \\

\yl{1} Snelli seadus (murdumisseadus) lähtuvalt Fermat' printsiibist.\\

\yl{2} Stefan-Boltzmanni seaduse tuletus Plancki musta keha kiirguse põhjal.\\

\yl{3} Wieni nihkeseaduse tuletus Plancki musta keha kiirguse seaduse põhjal.\\

\yl{4} Erinevate sagedustega lainete liitumine.\\



\end{document}
