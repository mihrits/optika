\documentclass[11pt]{article}

\usepackage[estonian]{babel}
\usepackage[utf8]{inputenc}
\usepackage{amsmath}
\usepackage{tikz}
\usepackage{graphicx, wrapfig}
\usepackage[per-mode=symbol, decimalsymbol=comma, exponent-product = \cdot]{siunitx}
\usepackage{enumerate}
\usepackage{physics}
\usepackage[margin=2.5cm]{geometry}
\usepackage{lmodern, microtype}
\usepackage{hyperref}

\setlength\parindent{0pt}

\newcommand{\yl}[1]{\noindent{\bf Ülesanne #1} \quad}
\newcommand{\lah}{\noindent{\bf Lahendus} \quad}

\begin{document}
{\small Optika (LOFY.01.008) seminar \hfill kevad 2020}\\

{\Large Kodutöö I}\\

Kodutöö lahenduste esitamine on vajalik kontrolltöödele pääsemiseks. Kui materjali lugedes või ülesandeid lahendades tekib küsimusi, siis pange need kirja ja esitage vastava osa õppejõule või enne seminari või seminaris seminaride läbiviijale Moorits Mihkel Murule (\texttt{moorits.mihkel.muru@ut.ee}).

\rule{\linewidth}{1pt} \\

\yl{1} Ülesande tekst\\

\yl{2} Ülesande tekst\\

\yl{3} Ülesande tekst\\

\yl{4} Ülesande tekst\\



\end{document}
